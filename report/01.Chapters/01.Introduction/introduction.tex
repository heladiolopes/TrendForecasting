Recently, the growth of artificial intelligence has been helping us to solve problems in the most various areas, including in linguistics. With natural language processing techniques, computers can process text in order to extract information faster than humans.

\section{Motivation}

Every kind of expression, verbal or in writing, brings us much information to be interpreted. Whether the topic is chosen, the tone used, the choice of words, everything can be interpreted, and then generate some valuable information. Over the years, more and more knowledge is generated and we humans are unable to process such an amount of information. Natural Language Processing emerges as a technology capable of assisting us in this hard task.

\citeonline{state_of_the_art} defines Natural Language Processing, abbreviated by NLP, as a branch of artificial intelligence capable of making computers comprehend and extract information from human language. NLP can perform a lot of tasks, such as identifying different topics for a set of documents, classifying texts on predefined subjects, and beyond that extract the sentiment to know what people are saying about something.

% 

\section{Objective}

Curious about the fast word's evolution, this work aims to implement Natural Language Processing techniques to propose a framework capable of modeling in real-time the topics' evolution over the years. With this in mind, evaluate the ability of those models to make predictions about future trends.

\section{Organization of this work}

The remaining of this work is organized as follows: Chapter \ref{chap:literature} will cover the theory behind Natural Language Processing. Chapter \ref{chap:related} will describe some previous works which use topic discovery and trend forecast. Chapter \ref{chap:materials} will better explain the problem and the methodology that will be used to handle it. Finally, Chapter \ref{chap:roadmap} will present the chronological roadmap until the conclusion of the work.