Recently, the growth of artificial intelligence has been helping us to solve problems in most various areas, including linguistics. With natural language processing techniques, computers can process text in order to extract information faster than humans.

\section{Motivation}

%Every kind of expression, verbal or in writing, brings us much information to be interpreted. Whether the topic is chosen, the tone used, the choice of words, everything can be interpreted, and then generate some valuable information. Over the years, more and more knowledge is generated and we humans are unable to process such an amount of information. Natural Language Processing emerges as a technology capable of assisting us in this hard task.

%\citeonline{state_of_the_art} defines Natural Language Processing, abbreviated by NLP, as a branch of artificial intelligence capable of making computers comprehend and extract information from human language. NLP can perform a lot of tasks, such as identifying different topics for a set of documents, classifying texts on predefined subjects, and beyond that extract the sentiment to know what people are saying about something.

%The topic discovery was widely explored in the literature, \citeonline{hurtado2016topic} and \citeonline{jelodar2020deep} use it to group, respectively, papers abstracts and comments in social media in similar clusters. In a long period, it is possible to analyze how these founded topics behave through time and make predictions about the near future.

Every kind of expression, verbal or in writing, brings us much information to be interpreted. Whether the topic is chosen, the tone used, the choice of words, everything can be interpreted, and then generate some valuable information.

Over the years, more and more knowledge is generated and we humans are unable to process such an amount of information. A study made by \citeonline{beath2012finding} reveals that the enterprises' volume of data is expanding by 35\%-50\% every year, including textual data. Despite this, they also find that the companies could not extract information from this huge amount of data.

In that way, extracting valuable information from this data has become more and more an important skill. Therefore, Natural Language Processing emerges as a technology capable of assisting us in the task of deal with textual data.

\citeonline{khurana2017natural} defines Natural Language Processing, abbreviated by NLP, as a branch of artificial intelligence capable of making computers comprehend and extract information from human language.
In fact, NLP can perform simple tasks but not quite scalable for a human, such as translate text from one language to another, identify different topics for a set of documents, classify text on predefined subjects, and beyond that extract the sentiment to know what people are saying about something.

% Falar sobre alexa, google assistente e siri.
In recent years a particular kind of NLP application is drawing a lot of attention. The voice assistants like Siri, Cortana, Alexa and Google Assistant are the hot AI topics of the moment. Together, all digital assistants attempted to answer more and more questions each year, \cite{kailakrayewski2019}.

\newpage
In our particular scope, the topic discovery was widely explored in the literature. \citeonline{hurtado2016topic} and \citeonline{jelodar2020deep} use it to group, respectively, papers abstracts and comments in social media in similar clusters. The first one uses it to study the topics' evolution and correlation between the founded topics. In contrast, the second group comments to analyze the general sentiment about them.

However, these related works apply topic modeling only for static analysis. In this work context, we are seeking a real-time application that will keep receiving news in a continuous process. Thus, redo the topic discovery process will demand an expensive computational cost, besides the human effort to label the topics. To avoid the rework of modeling the documents in topics, a classification-based system will be proposed, for a future development of a real-time application is possible.


\section{Objective}

%Curious about the fast world's evolution, this work aims to implement Natural Language Processing techniques to propose a framework capable of modeling in real-time the topics' evolution over the years. With this in mind, evaluate the ability of those models to make predictions about future trends.

Curious about the fast world's evolution, this work aims to implement Natural Language Processing techniques to propose a framework capable of modeling in real-time the topics' evolution over time, using an annual granularity. 
With this in mind, we must evaluate the ability of those models in modeling the themes occurrence over the years.

By the end of this work, a series of tasks must be done. First, we must normalize the documents to segment them into time-ordered subsets, \textit{Past}, \textit{Present} and \textit{Future}, respectively. With the eldest subset, we must find meaningful subjects to train classification models based on these topics. Then, evaluate the predictions for \textit{Present} set and how the models' performance degenerates over the years.

% Esta seção deve conter:
%   - Objetivo geral: propor um framework para aprender evolução de tópicos em texto em
%     linguagem natural ao longo do tempo.
%   - Objetivos específicos:
%     a) Propor a divisão da base de dados em diferentes conjuntos
%     b) Encontrar os tópicos no passado
%     c) Treinar classificadores
%     d) Avaliar o desempenho no presente
%     e) Avaliar a degeneração do desempenho

%\section{Objective}
%
%As already discussed, we want to build models capable of making predictions regarding the evolution of discovered topics in a set of documents. We also want to find topics in real-time at each received document without having to redo the topic discovery process. Then, by the end of this work, we must have performed the tasks listed below.
%
%\begin{itemize}
%	\item Find a database long enough, over several years;
%	\item Perform all necessary treatment steps to normalize the documents' texts;
%	\item Find meaningful topics in a subset from the original database;
%	\item Create a topic classifier to find out if a document addresses any topic of interest;
%	\item Model topic evolution to evaluate the forecast accuracy.
%\end{itemize}

\section{Organization of this work}

The remaining of this work is organized as follows: Chapter \ref{chap:literature} will cover the theory behind Natural Language Processing and basics Classification. Chapter \ref{chap:related} will describe some previous works which use topic discovery and trend forecast. Chapter \ref{chap:materials} will better explain the problem and the methodology that will be used to handle it. Chapter \ref{chap:results} will show the discovered results for the proposed experiment and some discussion about them. Finally, Chapter \ref{chap:conclusion} will present some conclusions and suggestions for future work.
