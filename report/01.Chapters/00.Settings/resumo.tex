Com avanço cientifico-tecnológico em escalas globais, a disponibilidade de dados para serem analisados cresce de maneira exponencial. A partir desse crescimento acelerado, técnicas de inteligência artificial, como o Processamento de Linguagem Natural (PLN), ajudam a automatizar o processo de se extrair informação de documentos.  Com intuito de realizar previsões sobre as tendências que surgirão em meio ao crescimento de dados, é possível utilizar técnicas de modelagem de tópicos para agrupar documentos em temas semelhantes e estudar a incidência temporal desses temas.

Assim, este trabalho busca aplicar técnicas de PLN combinadas com aprendizado supervisionado para agrupar documentos em um conjunto de temas e a partir disso aprender a identificá-los em novos documentos em um período futuro. Utilizando publicações cientificas da conferência de Sistemas de Processamento de Informação Neural, fomos capazes de modelar essa tarefa e obter bons resultados. Também estudamos a capacidade dos classificadores de manter desempenho satisfatório com o passar do tempo.
